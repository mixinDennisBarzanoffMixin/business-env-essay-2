%% This is annot.tex.
%% 
%% You'll need to change the title and author fields to reflect your
%% information.
%%
%% Author: Titus Barik (titus@barik.net)
%% Homepage: http://www.barik.net/sw/ieee/
%% Reference: http://www.ctan.org/tex-archive/info/simplified-latex/

\documentclass{article}

\usepackage[T1]{fontenc}
\usepackage[utf8]{inputenc}
\usepackage{mathptmx}
\usepackage{setspace}


\usepackage[
    style=authoryear,
    sorting=none
]{biblatex}
\usepackage{lipsum}% just to generate text for the example
\usepackage{hyperref}
\usepackage{dingbat}
\usepackage{array}
\usepackage{booktabs,adjustbox}

\fontsize{11pt}{11pt}\selectfont

\hypersetup{
    colorlinks=true,
    linkcolor=blue,
    filecolor=blue,      
    urlcolor=blue,
    citecolor=blue,
    pdftitle={The Impact Which Technology Has Upon the Accounting Profession},
    pdfpagemode=FullScreen,
}

\urlstyle{same}



\addbibresource{annot.bib}



\title{The Impact Which Technology Has Upon the Accounting Profession
\\\medskip Essay}
\author{
    \begin{tabular}{@{}llr}
        Barzanoff   & Dennis        & d.rumenoffbarzanoff@lancaster.ac.uk\\
    \end{tabular}
}



\begin{document}
\onehalfspacing
\maketitle

\section*{Introduction}
Accounting is a profession that deals with the structuring of data that can later be used for making financial decisions in the future. 
Due to the rapid developments in technology over the past years, computer software has become much better at doing certain things than humans, 
particularly those with easy patterns to spot. It has also become a very useful tool for the accounting profession.


\section*{Main}
This rapid development, however, has widened the gap between what accountants are taught at school and what the real-world market actually needs. 
This trend has caused the emergence of organisations that deal with this matter, such as SAICA. \parencite[][]{EloffAnne-Marie2016Tioi}. 
That means people are struggling to keep up with all the new software that is being developed. 

That does not mean however that  people do not see the benefits of using technology to ease their work. In fact most students prefer using programs 
rather than pen and paper for their work. Electronic formats provide comfort and according to \parencite[][]{EloffAnne-Marie2016Tioi}. 
72\% of students learning accounting are more motivated to do the task if it is online, giving their practical experience as proof. 
That is why most accountants nowadays make a lot of use of professional software like Microsoft Excel to help organise information and speed up calculations. \\

Email and the internet are only some of the tools used by people in this area. \parencite[][p. 42]{OladeleFemi2020SMMa} They facilitate the problem of having too much information 
(\textbf{overload}), indexing it (Big data) and tools like technology assisted accounting have made manual bookkeeping redundant.
Use of technology in accounting comes with its problems, one of which is \textbf{overtrust}, which can happen when an organisation implements a software program 
into its routine without prior information whether this piece of software is capable of adequately solving the problem it was used for and realising later 
during the process that this software is not appropriate for this application. \parencite[][p. 42-43]{OladeleFemi2020SMMa}
Another risk of adopting technology in accounting is the fact that accountants have to \textbf{trust} the software they are using that it is going to do the job 
it is supposed to do and nothing else. It must not share data with external sources, however, the code of most applications is closed source, meaning that 
it is extremely difficult for one to uncover exactly what the program is doing at any given moment.

According to a study \parencite{AbdelraheemAbubkrAhmedElhadi2021Teoi} on the effect of IT on accounting, usage of technology in accounting has an influence over 
the quality of the accounting records output. This article makes a great resource because it makes use of statistics to provide the user with matematical proof of their the findings. 
The study shows that it increases “relevance, reliability, understandability, consistency and comparability 
of accounting information, and consistency in processing and displaying accounting information”.

Ultimately accounting information is there to provide one with a summary of the insight of the internals of an organisation. If the information is not good, 
then the insight it gives is not good either. \parencite{AbdelraheemAbubkrAhmedElhadi2021Teoi}


One of the more recent employments of technology in accounting has been in the area of \textbf{auditing}. \parencite{PetkovRossen2020AiAa}, making financial decisions and 
generating financial statements, the last of which is something the company Cisco is doing. To generate financial statements, artificial intelligence needs data 
about property, plant, equipment, inventory, etc. and to make certain predictions to calculate the depreciation, obsolete inventory and so on.
It also takes less \textbf{time} to prepare a financial statement when using AI.
The statements that are program generated are less prone to \textbf{mistakes}. 

Another risk that technology poses to many jobs, not just accounting, is that it may lead to unemployment, at least in the early stage of development \parencite{MirzaeiAbbasabadiHamed2021Eteo}

\section*{Conclusion}
Using technology to facilitate operations of accountants can be very beneficial in reducing costs of a company in the long run and increasing reliability of the records produced. 
That said, one needs to be cautious of all of the downsides of using tech, such as using the wrong tool for the job. All things considered, there are much more upsides to using technology than downsides, 
but one should still be mindful when adopting such systems into their jobs. \\

\section*{Misc}
Word Count: 700\\
I know the referencing fields are not separated by comma, but to do I would have had to reconfigure the entire bib parser, so I would rather take a couple of points for now, I'm sorry.
I did try to fix the formatting this time tho.

\printbibliography

\end{document}