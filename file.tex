%% This is annot.tex.
%% 
%% You'll need to change the title and author fields to reflect your
%% information.
%%
%% Author: Titus Barik (titus@barik.net)
%% Homepage: http://www.barik.net/sw/ieee/
%% Reference: http://www.ctan.org/tex-archive/info/simplified-latex/

\documentclass [11pt]{article}
\usepackage[
    style=authoryear,
    sorting=none
]{biblatex}
\usepackage{lipsum}% just to generate text for the example

\addbibresource{annot.bib}



\title{Threats and Opportunities to the Business Sector From
Increasing Economic Protectionism.\\\medskip An Annotated Bibliography}
\author{
    \begin{tabular}{@{}llr}
        Barzanoff   & Dennis        & d.rumenoffbarzanoff@lancaster.ac.uk\\
        Biro        & Daniel        & d.biro@lancaster.ac.uk\\
        Venkatesh   & Vikrant       & v.venkatesh1@lancaster.ac.uk\\
    \end{tabular}
}



\begin{document}
\maketitle

\begin{abstract}
Economic stability depends on many factors, one of which is trading. A country may choose to limit or free its trading
as a way of aiding its development and economic stability. A country may choose to limit its trade in order to subsidise its local industries. 
That is counter intuitive because in most cases more trade means more competition, which reduces prices and brings more diversity to consumers.
On the other hand this suppresses small businesses as they can hardly compete with already-established ones directly, putting small industies at a
disadvantage.
\end{abstract}

\section{Journal}

The author argues that despite the fact that free trade increases the economy of a country, it is not necessarily 
beneficial for the people, because in some cases it leads to job destruction and migrations. In the newspaper article there are manifold examples 
concerning the impact of free trade on most developing African countries and, more specifically, the consequences of reducing government support for 
infant industries. \textcite{2018TIGP} claims that job destruction, unemployment and migration are among some of the repercussions that follow when a
country is to remove tariff barriers and other methods of protection for small businesses. 
\parencite[][p.5]{2018TIGP}


One noble example is Botswana. It used to have a flourishing diamond industry, but it was hampered by the free market and has become a predominantly export-oriented industry due to
the decrease in cost caused by free trading. This has led to the imminent destruction of many skillful jobs that require time to create. 
The newspaper also compares recent actions concerning the USA and the rules enforced by the right-winged political party and suggests that if America is to open its economy and remove the
tarrifs, it risks having the same grim faith as the African market. 
\parencite[][p.6]{2018TIGP}

\section{Book}

\lipsum[1-2]

\section{Magazine}

\lipsum[3-4]

\section*{Conclusion}
Enforcing the correct measures to help economical growth is not always obvious. Reliance on other economies and independence coexist in a delicate balance that ultimately leads to economic stability. 
Free trade spawns some temporary growth, but that comes with a cost that is hard to recover from.
\printbibliography

\end{document}