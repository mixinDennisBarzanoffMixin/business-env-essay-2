%% This is annot.tex.
%% 
%% You'll need to change the title and author fields to reflect your
%% information.
%%
%% Author: Titus Barik (titus@barik.net)
%% Homepage: http://www.barik.net/sw/ieee/
%% Reference: http://www.ctan.org/tex-archive/info/simplified-latex/

\documentclass [11pt]{article}
\usepackage[
    style=authoryear,
    sorting=none
]{biblatex}
\usepackage{lipsum}% just to generate text for the example
\usepackage{hyperref}
\usepackage{dingbat}
\usepackage{array}
\usepackage{booktabs,adjustbox}


\hypersetup{
    colorlinks=true,
    linkcolor=blue,
    filecolor=blue,      
    urlcolor=blue,
    citecolor=blue,
    pdftitle={Overleaf Example},
    pdfpagemode=FullScreen,
}

\urlstyle{same}



\addbibresource{annot.bib}



\title{Threats and Opportunities to the Business Sector From
Increasing Economic Protectionism.\\\medskip An Annotated Bibliography}
\author{
    \begin{tabular}{@{}llr}
        Barzanoff   & Dennis        & d.rumenoffbarzanoff@lancaster.ac.uk\\
        Biro        & Daniel        & d.biro@lancaster.ac.uk\\
        Venkatesh   & Vikrant       & v.venkatesh1@lancaster.ac.uk\\
    \end{tabular}
}



\begin{document}
\maketitle

\section*{Introduction}
Economic stability depends on many factors, one of which is trading. A country may choose to limit or free its trading
as a way of aiding its development and economic stability. A country may choose to limit its trade in order to subsidise its local industries. 
That is counter intuitive because in most cases more trade means more competition, which reduces prices and brings more diversity to consumers.
On the other hand this suppresses small businesses as they can hardly compete with already-established ones directly, putting small industies at a
disadvantage.

\section{Journal}

The author argues that despite the fact that free trade increases the economy of a country, it is not necessarily 
beneficial for the people because in some cases it leads to job destruction and migrations. In the newspaper article there are manifold examples 
concerning the impact of free trade on most developing African countries and, more specifically, the consequences of reducing government support for 
infant industries. \textcite{2018TIGP} claims that job destruction, unemployment and migration are among some of the repercussions that follow when a
country is to remove tariff barriers and other methods of protection for small businesses. 
\parencite[][p.5]{2018TIGP}


One example, viciously defended by the author, is Botswana. According to the journal, the country would have developed a flourishing diamond industry, had it not been hampered by the free market because of which it has become predominantly export oriented due to
the decrease in cost caused by free trading. This has led to the imminent destruction of many skillful jobs that require time to create. 
The newspaper also compares recent actions concerning the USA and the rules enforced by the right-winged political party and suggests that if America is to open its economy and remove the
tarrifs, it risks having the same grim faith as the African market. 
\parencite[][p.6]{2018TIGP}

\section{Book Report}

The report strongly points out of how protectionism of maritime trade has reduced the benefits to national economies. The report outlines that if the protectionist trade policies implemented by the governments worldwide are cut then the GDP of would significantly increase for those countries.  
\parencite{CraigVang2021PiMES}

They have given four scenarios which can be used to increase the GDP of national economies if protectionism is removed or cut and how the benefit depends on the country. The scenarios show that if the country is highly ambitious, they can increase their trade of goods and services which gives significant gain in economy. It also shows that modest and equal and unequal ambition will still gain significant rise. 

The study shows how low/middle income countries should be modest and equally ambitious in cutting their protectionist policies which would allow them to gain the most gain in their GDP. By contrast, for the high-income countries the different effects of these two scenarios would be modest. The data reveals that, no matter what their level of economic development, all countries would benefit from reduction and removal of protectionism even if they do so individually. 

The author hopes the governments of these countries use the report to come back from the negative impact that the COVID-19 had on the economy due to the protectionism policies as the study gives rational ways in which countries can increase economic growth in difficult times 

\section{Magazine}

The article cited presents a strong view over the current position of the Ontario job market, and the policies that the government is planning to implement in order to make the situation better off for the Canadian population. However, the author takes a critical position in regards to the proposed solutions, judging and explaining everything from an economic perspective. 

We can clearly observe that there is no gain in welfare generated simply by applying the principle of protectionism in regards to the job market because such measures would actually cut off a big part of the participants in the economy \parencite{MatthewLau2020Adep}. It is argued that, even though the pay checks do not go to Canadian citizens, but to the immigrant workforce, that money is still spent and invested mainly inside of Canada’s borders, so it helps the process of creating more jobs for the Canadians as well. 

The author also gives an example of when these measures were taken recently, and the lack of effectiveness they presented. Evidently, the example used is president Trump’s attempt to reduce the Mexican workforce in favor of the US one. As everybody knows by now, the results were not as spectacular as hoped by the head of state, so there is no reason to believe that in Canada’s situation things would present a different outcome. 

Another point of view presented is the interventionist one, which states that there is actual economic growth to be had if the government is taking an active stance, and telling the population and businesses what to buy and who to hire \parencite{MatthewLau2020Adep}. This way of thinking can be easily combated with having some economic principles in mind when judging it. 


% \nocite{*}

\section*{Conclusion}
Reliance on other economies and independence coexist in a delicate balance that leads to economic stability. Our research highlights the fact that in some cases protectionist policies can be helpful for the economic situation of a country. Restrictions applied to the companies originating outside the country in question would result in economic growth solely based on the lack of competition from the indigenous companies.  

It is not necessary the best policy if the welfare of the population is a big concern, the lack of competition in the market generating no need for the prices to be competitive, and thus companies would just take care of their own interest, that being profit. Opening the market, however, has its own risks, such as destruction of jobs due to the increase of imports for products. 

In the long term we can observe an interesting phenomenon regarding the potential upside generated globally by a free circulating market. We classify this effect as interesting for the whole reason of how counter intuitive it is when observing the situation on a state-by-state basis.  
\printbibliography

\end{document}